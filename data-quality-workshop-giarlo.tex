\documentclass[man,12pt,biblatex]{apa6}
\usepackage[american]{babel}
\usepackage{csquotes}
\usepackage{hyperref}
\usepackage[T1]{fontenc}
\usepackage{endnotes}
\let\footnote=\endnote
\DeclareLanguageMapping{american}{american-apa}
\addbibresource{data-quality-workshop-giarlo.bib}

\title{Academic Libraries as Data Quality Hubs}
\shorttitle{Data Quality Hubs}
\author{Michael J. Giarlo}
\leftheader{Giarlo}
\affiliation{Penn State University}
\abstract{Academic libraries have a critical role to play as data
  quality hubs on campus, based on the need for increased data quality
  to sustain ``e-science,'' and on academic libraries' record of
  providing curation and preservation services as part of their
  mission to provide enduring access to cultural heritage and to
  support scholarly communication. Scientific data is shown to be
  sufficiently at risk to demonstrate a clear niche for such services
  to be provided. Data quality measurements are defined, and digital
  curation processes are explained and mapped to these measurements in
  order to establish that academic libraries already have sufficient
  competencies ``in-house'' to provide data quality
  services. Opportunities for improvement and challenges are
  identified as areas that are fruitful for future research and
  exploration.}
\keywords{data quality, digital curation, digital preservation,
  academic libraries, stewardship, e-science, research data, trust}
\authornote{michael@psu.edu}

\begin{document}
\maketitle
\section{Introduction}
Academic libraries have a critical role to play as data quality hubs
on campus, providing data quality services for the research
enterprise. In order to sustain ``e-science'' or ``e-research,'' an
emerging paradigm for scientific practice that relies upon data reuse,
researchers need to be sure that the data they are using is
high-quality data. The data that will enable the e-science paradigm is
at risk for numerous reasons, such as its size, the rate of its
growth, its heterogeneity, and the lack of archival storage underlying
it. There is a growing niche for data quality services.

Data quality measurements are examined to determine a common set of
salient quality indicators, which are then examined vis-\`{a}-vis
existing digital curation practices. Digital preservation and
curation, emerging but nonetheless core competencies of academic
libraries, are argued to be applicable to needed data quality
services, by mapping curatorial practices to established data quality
measurements.

Academic libraries have a record of offering curation and preservation
services as part of their mission to provide enduring access to
cultural heritage and to support scholarly communication.  In short,
academic libraries have both the wherewithal and the mission to
intervene in a critical area on campus. Opportunities, both practical
and aspirational, and challenges are identified as areas that are
fruitful for future research and exploration. Academic libraries are
ready for the challenge.

\section{Research Data at Risk}
Data quality is a pressing, not to mention costly, issue in industry;
a 2002 study \parencite{russom:case} calculated that over \$600 billion per
year was spent on ``data quality problems''\ 
\parencite{eckerson:bottomline}.  At the same time, data quality issues
have become an area of growing attention within academia and academic
libraries \parencite{heidorn:libraries,arl:stewardship,ogburn:imperative,jisc:deluge},
as scientific practices evolve to exploit robust campus
cyberinfrastructure and as funding agencies, such as the National
Science Foundation and the National Institutes of Health, increasingly
require data management plans to protect and amplify the impact of
their investments.

As computing costs have dwindled, computer processing speed, network
throughput, and storage capacity have grown, resulting in an explosion
of research data.  It is not uncommon for experiments, in some
disciplines more than others, to produce more data than their
principal investigators and research assistants can
handle \parencite{adams:galaxyzoo}. Due to the wealth of data that is
being produced, scientific practice is changing; the gathering of data
for one experiment may drive dozens or hundreds of other experiments
around the world \parencite{jisc:deluge}.

Research data is more abundant, and no less important, than ever
before. Much of this data is deposited into disciplinary repositories
and data banks that are funded by the
U.S. government\parencite{merali:peril,baker:funding}. The government,
however, has shown a preference for funding new services over
maintaining existing services that jeopardizes the future of
disciplinary data repositories \parencite{merali:peril}. The 2008-2012
global recession has exacerbated the funding crisis for such
repositories \parencite{baker:funding}.

The long-term stewardship of research data is thus at
risk \parencite{ogburn:imperative}.  In light of the funding crises
faced by government-supported disciplinary data repositories, which
threatens the availability of research data, there is a niche for
centrally-funded organizations such as academic libraries and academic
information technology to provide for the stewardship of research
data. ``The survival of this data is in question since the data are
not housed in long-lived institutions such as libraries. This
situation threatens the underlying principles of scientific
replicability since in many cases data cannot readily be collected
again''
\parencite{heidorn:libraries}.

There are numerous examples in the literature of analog data enabling
scientific inquiry decades and longer past the date it was gathered
\endnote{\textcite{ogburn:imperative} cites Stephen Jay Gould's The
  Mismeasure of Man: ``analysis and critique of cranial measurements
  in the 1800s, twin studies in the 1950s, and the rise of IQ testing
  were possible because the data were still available for scrutiny and
  replication''}; how do we as a society, and particularly we within
academia, not only preserve this wealth of data for future science but
ensure it is of high quality?

\subsection{Curatorial Practice and Challenges}

Cultural heritage organizations such as libraries and archives have
been stewards of society's cultural and scientific assets for
millennia, providing public access to high-quality collections, and
they remain so in the Internet age. Though the activities involved are
different for analog assets, ``[s]tewardship of digital resources
involves both preservation and curation. Preservation entails
standards-based, active management practices that guide data
throughout the research life cycle, as well as ensure the long-term
usability of these digital resources. Curation involves ways of
organizing, displaying, and repurposing preserved data''
\parencite{arl:stewardship}.

Digital preservation and digital curation, though relatively new
practices, are widely treated in the literature
\parencite{jisc:deluge,curry:community,goble:curation,ogburn:imperative,heidorn:libraries,williams:lifecycle,arl:stewardship}. Digital
curation aims to make selected data accessible, usable, and useful
throughout its lifecycle. Digital curation subsumes digital
preservation; without viable data, which digital preservation enables,
there's nothing to be curated \footnote{This characterization of
  digital curation and digital preservation is a mere gloss; more may
  be found, for instance, on the Digital Curation Centre's website:
  \url{http://www.dcc.ac.uk/digital-curation}.}.

An oft-cited mantra on the practice of digital curation is that
``curation begins before creation [of the data]''
\parencite{rusbridge:curation}. And yet, ``[b]y the time knowledge in
digital form makes its way to a safe and sustainable repository [such
as those provided by academic libraries], it may be unreadable,
corrupted, erased, or otherwise impossible to recover and
use. Scientific data files may be especially endangered due to their
sheer size, computational elements, reliance on and integration with
software, associated visualizations, few or competing standards,
distributed ownership, dispersed storage, inaccessibility, lack of
documented provenance, complex and dynamic nature, and the concomitant
need for a specialized knowledge base --- and experience --- to handle
data.  Data also may be endangered by the practices of scholars who
regard their data as having little value beyond the confines of a
small group, a specific project, or a specified period''
\parencite{ogburn:imperative}.

\subsubsection{\textit{Post-Hoc Curation Considered...}}

As digital curation is a new practice, and is generally centered
within cultural heritage organizations \footnote{The work of
  discipline-specific repositories such as \textit{e.g.}, the Protein Data Bank,
  GenBank, the Biomedical Magnetic Resonance Data Bank, Dryad, and others
  are notable exceptions.}, \textit{post-hoc} curation is an unfortunate
fact of life; researchers lack the incentive, the resources, the time,
or the expertise to curate their own data \footnote{Hereafter referred
  to as ``sheer curation or curation at source''
  \parencite{curry:community}.}, and so its curation falls to other parties
after the data has been created, and often after it has been
``archived.'' For especially massive data sets, furthermore, it is difficult
even to imagine, \textit{e.g.}, a research institute or academic department
having sufficient resources to curate their own data at scale.

The practice of \textit{post-hoc} curation (vs. ``sheer curation,'' or
curation by researchers at the time of creation) is less than ideal
for a number of reasons.

First, one of the goals of curation is to enable the usefulness of a
digital resource over time, and one of the tactics applied is to
provide sufficient context for a resource such that future users can
understand what an object is, where it came from, why it is
significant, and how to use it. Context is often provided via
documentation, descriptive metadata, or both
\parencite{arl:stewardship,heidorn:libraries,curry:community,jisc:deluge}. The
creator(s) of the data, \textbf{not} its \textit{post-hoc} curators,
are best equipped to provide this context; to get a sense of this
distinction, consider the difference between the tasks of cataloging
your own book collection and cataloging a complete stranger's book
collection.

Second, building on the prior reason, is that \textit{post-hoc}
curation happens some time after the data have been created, possibly
a long enough time to lose track of important information; capturing
the context around a data set is best done while the data is still
fresh in its creator's mind, \textit{i.e.}, before or during its
creation. Documentation or metadata that is created by a party other
than the data's creator, especially when performed after the
responsible parties have moved on to other challenges, will suffer
from this lack of context.

``This [\textit{post-hoc} curation] activity is to provide
representational information and description. This is particularly
problematic for academic libraries, since the data being generated at
research and teaching institutions are incredibly varied. Many
representational schemes for the data and metadata will be
required. No one individual will have all of the required skills. Data
curators will need to collaborate closely with the data providers to
understand the data'' \parencite{heidorn:libraries}. Whether researchers
will have sufficient time, resources, and inclination to collaborate
with academic libraries on the work of curating research data at scale
is yet to be seen.

Finally, possibly the most limiting reason: there is a misalignment
between the scale of the need for on-campus data curation and the
level of commitment by academic libraries to address this need (as
measured by the amount of resources allocated to this need vs. other
needs). Data curation efforts are often understaffed and
underresourced, with many academic libraries devoting one full-time
equivalent employee, if that, to this role, to say nothing of the
level of administrative and staff support for this role.

Academic libraries, institutional will and administrative support
notwithstanding, are nonetheless uniquely positioned to tackle the
problem of data quality in e-science by virtue of their record of
effective stewardship, their commitment to providing access to
high-quality data over the long-term, and their expertise in digital
preservation and digital curation practices, as ``[digital] curation
is a process that can ensure the quality of data and its fitness for
use'' \parencite{curry:community}. It is worth examining this claim in the
context of a framework for measuring data quality.

% This stuff doesn't seem to fit any longer
%
% While cyberinfrastructure has become more robust, usage of the
% Internet continued to grow; between 2000 and 2010 alone, Internet
% usage rose from nearly 7\% of the global population to over 30\%
% \footnote{Data visualized here:\url{http://www.google.com/publicdata/explore?ds=d5bncppjof8f9_&met_y=it_net_user_p2&tdim=true&dl=en&hl=en&q=global+internet+usage}}.
% trust networks - how are they different now? japanese research example
% (more numerous and more anonymous) \parencite{timmer:faking}, galaxyzoo
% example (much larger and more participatory (citizen
% science)) \parencite{adams:galaxyzoo}.
%
% refer to \parencite{timmer:faking} issue. hacking trust networks, new trust
% networks, more resilient trust networks -- look into identity
% assurance (concept from federated identity and access management area)
% and assurance levels, specifically applying it to data authenticity
% (now that researcher ID systems such as ORCID or ISNI are gaining
% steam, perhaps requiring more than signatures (OAuth via ORCID, etc.)
% could help, as could karma-like systems) ``assurance levels'' as relevant
% concepts?

\section{Measuring Data Quality}
There are a number of theoretical frameworks quantifying data quality
measures already established, and Knight's 2005 paper compares a
selection of a dozen ``widely accepted [information quality]
Frameworks collated from the last decade of [information science]
research'' \parencite{knight:quality}. Common features are identified for
data quality (or information quality), such as that it is a concept
with multiple dimensions, wherein the overall quality is a function of
successive indicators.  Another common feature of data quality
frameworks is the grouping of quality indicators into categories,
classes, or levels corresponding to, \textit{e.g.}, semiotic levels,
layers of intrinsicity and extrinsicity, and the subjectivity /
objectivity spectrum.

The following framework is distilled from Knight's comparison of
quality frameworks, and constitutes ``a series of quality dimensions
which represent a set of desirable characteristics for an information
resource'' \parencite{curry:community}. The framework is then applied to
the domain of research data quality as viewed from my perspective,
that of a digital preservation technologist and practitioner of
digital curation. It is not offered as a novel framework, nor a
comprehensive one, but merely as a tool for understanding and
evaluating the applicability of digital curation and preservation
practices to the measure of data quality.

\begin{APAitemize}
\item \textbf{Trust}: Evaluation of the extent to which data is
  trusted depends on a set of subjective factors, including whether
  the data is judged to be authentic, the uses to which the data is
  put, the subject discipline, the reputation of the party/ies
  responsible for the data, and the biases of the person who is
  evaluating the data \footnote{Trust is a complex issue that though
    relevant is too far-reaching to be within the the scope of this
    position paper. It is nonetheless listed in the framework at the
    very top to establish that lower layers may be entirely discounted
    by an individual judging data quality if there are overriding
    trust issues. This topic is fertile for subsequent research}.
\item \textbf{Authenticity}: Evaluation of the authenticity of data
  requires that data be understood. Authenticity in this context is a
  rough measure of the extent to which the data is judged to be ``good
  science,'' answering questions pertaining to, \textit{e.g.}, the
  reliability of the instruments used to gather the data; the
  soundness of underlying theoretical frameworks; the completeness,
  accuracy, and validity of the data; and ontological consistency
  within the data.
\item \textbf{Understandability}: Evaluation of the understandability
  of data requires that there be sufficient context (documentation,
  metadata, or provenance) describing the data, and that the data is
  usable.
\item \textbf{Usability}: Usability of data requires that data is
  discoverable and accessible; that data is in a usable file format;
  that the individual judging the data's quality has an appropriate
  tool to access the data; and that the data is of sufficient
  integrity to be rendered.
\item \textbf{Integrity}: Integrity of data assumes that the data can
  be proven to be identical, at the bit level, to some prior accepted
  or verified state. Data integrity may be required for usability,
  understandability, authenticity, trust, and thus overall quality,
  though this depends in part of the level of perturbation of
  integrity. Integrity changes will have varying effects depending on
  how significant the perturbation is, the file format, and where
  within the file the perturbation has occurred.
\end{APAitemize}

The relationship between the quality dimensions in this framework is
analogous to that of the Semantic Web Layer Cake in that ``each layer
exploits and uses capabilities of the layers below''
\parencite{wiki:semweb}. Viewed from the bottom up, this framework asserts
that data integrity may be necessary but not sufficient for data
quality; if the data lacks integrity, it may not be usable, and thus
not understandable, authentic, or trustable --- a very low measure of
quality. On the other hand, unauthorized changes at the bit level may
not effect the rendered data in any perceivable ways. Viewed from the
top down, on the other hand, if an individual trusts a data set, she
likely judges it to be of the highest quality even if it is not
usable, understandable, or fixed in integrity.

\section{Applying Curation to Data Quality}

Within the defined framework, how might the practice of curation help
ensure data quality? Each of the indicators in this framework is
evaluated within the context of the digital curation lifecycle
\parencite{dcc:lifecycle}.

\subsection{Integrity}

The curation lifecycle contains actions geared towards preservation of
the digital asset, which includes bit-preservation via a number of
possible tactics such as regular digital signature (or checksum)
verification, replication, media refreshing, version management, and
file-level backups. These tactics taken together should be sufficient
to ensure that the data remains in the same state as originally
processed. Assuming that the data was authentic to begin with
\footnote{Authenticity is evaluated higher up the stack.}, the
effective practice of curation should provide data integrity.

\subsection{Usability}

Three of the seven sequential actions defined in the lifecycle model
have a direct impact on the usability of data. First, the Create or
Receive action \footnote{Again underscoring the mantra that ``curation
  begins before creation''} should include determination of an
appropriate file format for the data, choosing a format that is judged
to be widely accessible and preservable. The Access, Use, \& Reuse
action ``[e]nsure[s] that data is accessible to both designated users
and reusers, on a day-to-day basis'', thus ensuring that the data is
discoverable and made available to potential users of data. The
Transform action, lastly, includes periodic evaluation of file formats
and migration to new formats so data remain usable well after the
original formats have been rendered obsolete.

\subsection{Understandability}

Context is provided for data, in order that users may understand the
data, both in sequential actions within the curation lifecycle ---
those being Create or Receive and Preservation Action --- and also
within the full lifecycle action of Description and Representation
Information. The generation, extraction, and application of metadata
by machine agents and humans is thus a key part of the curation
lifecycle, providing periodic management and addition of context to
data.  These actions make sure the data's purpose, impact, and
provenance are established over the course of its lifecycle so that
current and future users can make sense of data that they have
discovered.

\subsection{Authenticity and Trust}

Authenticity and trust as dimensions of data quality are highly
subjective.  The curation process can document what instruments are
used to generate data, but not how reliable a user judges those
instruments to be; it can include metadata about the theoretical
frameworks underlying the data, but not whether the frameworks are
theoretically sound; it can clearly establish the parameters of the
data, but it is up to the user to judge whether those are a complete
or incomplete set of parameters. The context, provenance, and
documentation provided by curation are thus critically important in
arming users of data with the information they need to make quality
judgments but are \textbf{not} capable of independently ensuring data
authenticity or trust in data; that is entirely for the individual
user to judge.

Digital curation practices within academic libraries are sufficient to
provide a high level of data quality to the research
enterprise. Existing services that have been built around these
practices could justifiably be marketed as data quality services on
campus. The term ``data quality hubs'' is thus not substantially
different from ``data curation hubs'' but rather frames the
competencies of academic libraries in a way that applies to the
emerging and yet critical area of e-science.

\section{Areas of Opportunity}

\subsection{Curation Models}
Given the issues with the practice of \textit{post-hoc} curation
raised above, it is worth examining alternative curation models. This
is not to suggest that one model of curation is to be selected
exclusively; a mix of \textit{post-hoc} curation and
curation-at-source models will likely be in place at most
institutions.

The work required for doing curation at the source needs to be
incentivized and integrated into the researcher's extant
workflows. Unless there are clear and valuable incentives for
researchers to spend time and thought on curatorial work, and unless
curation can be made to fit into the way researchers currently work,
curation will be an after-thought, and thus so will data quality.

These different curatorial models are not mutually exclusive and in
fact it may be ideal to combine them, leveraging both the researcher's
deep domain knowledge and the professional curator's commitment,
expertise, and tools to preserve data quality over time.

\subsubsection{Scaling \textit{Post-Hoc} Curation}

Curry has examined a number of successful community-based curation
models, which may offer academic libraries a way to scale
\textit{post-hoc} curation and deal with the aforementioned
deficiencies of this approach: ``[d]ata curation teams have found it
difficult to scale the traditional [\textit{post-hoc} curation]
approach and have tapped into community crowd-sourcing and automated
and semi-automated curation algorithms'' \parencite{curry:community}.

The rise of the ``citizen science'' paradigm, such as demonstrated in
the Galaxy Zoo and Zooniverse projects
\parencite{wiki:galaxyzoo,adams:galaxyzoo}, suggests community
crowd-sourcing as a tactic that may be used to complement an
institution's curation model. These initiatives leverage the ``wisdom
of the crowd'' in curating \footnote{Or, at least, classifying,
  cataloging, and otherwise annotating these data sets, even if it not
  inclusive of all activities within the curation lifecycle.}  massive
data sets such as the astronomical image data in the original Galaxy
Zoo project. Galaxy Zoo in particular has been wildly successful,
attracting a user base numbering into the hundreds of thousands, who
have worked together to classify hundreds of millions of records
\parencite{adams:galaxyzoo}.

There are numerous incentives at play in crowdsourcing, such as access
to broadly interesting and compellingly visualized data; competition;
and a desire for the layperson to be involved with \textit{bona fide}
research with opportunities to make novel scientific discoveries
despite limited domain expertise. Consider ``Hanny's
\textit{Voorwerp} \parencite{wiki:voorwerp},'' an astronomical body
discovered in Galaxy Zoo's data set by an amateur astronomer. The
\textit{Voorwerp} is now being studied by more than one professional
astronomer, studies that may never have happened if not for the
serendipitous discovery of an untrained curator.  There are numerous
other collaborative or crowd-sourced curation efforts highlighted in
Curry's chapter on community data curation \parencite{curry:community}.

Galaxy Zoo and other Zooniverse projects demonstrate aspects of a
model that could be repurposed in academic libraries as libraries seek
alternative models for research data curation that scale out.

As mentioned earlier, some combination of \textit{post-hoc} curation
and curation-at-source seems effective. The Galaxy Zoo project
balances crowd-sourced curation with verification by trained
astronomers \parencite{adams:galaxyzoo}, who verify samples of curatorial
work over time, thus enabling network effects to take place --- this
form of training or correction is not unlike the balance between human
correction and machine learning algorithms, or, \textit{e.g.}, the
reCAPTCHA \footnote{\url{http://www.google.com/recaptcha}}
service. This sort of delegation of quality to the community is not
unlike a principle found in the open source software world, which is
that the more eyes are on a codebase, the more likely it is that
defects will be found and corrected.

The challenges that face academic libraries in leveraging
crowd-sourcing as part of an institutional data curation strategy,
each of which bears more in-depth consideration or research, are
finding or allocating sufficient resources to build tools; finding
effective incentives to curate research data; building a community
around the data that is large enough to realize the benefits of
network effects; and coming up with a model that puts the ``trust but
verify'' strategy, whereby a sampling of crowd-curated records is
checked for quality (and corrected if need be), into effect at scale.

\textcite{curry:community} has identified a number of social and
technical best practices around community curation, which may be
useful in addressing these challenges: early and sustained stakeholder
involvement; outreach beyond the existing community via multiple
channels including both emerging social media and more traditional
channels such as newsletters and mass email; connection of curation
activities to tangible payoffs; an appropriate and clear governance
model; community-standard data representations; balance between
automated and human curation with the latter always overriding the
former; and recording and displaying provenance events to provide
additional context to crowd curators and users.

In addition to human curation, whether via trained curators or citizen
curators in ``the crowd,'' there is a growing number of increasingly
sophisticated tools for automated curation which could be used as a
less costly and more timely tier of curation (until such time as a
human curator has time to curate a data set). Tools for automated
curation such as for subject classification, part-of-speech tagging,
semantic entity extraction, and characterization can provide
much-needed context to enable some level of understandability,
usability, authenticity, and trust. Automated curation can thus help
with data quality in a way that scales in a less resource-constrained
way than requiring intensive human curation of every data set.

\section{Conclusion}

Academic libraries have an opportunity to serve as data quality hubs
on campus, extending their established digital curation and
preservation services to the research enterprise, doing for e-science
what libraries have a wealth of experience doing for other areas of
scholarly communication. With the scramble to establish data
management support services in the wake of the NSF's data management
plan requirement, the timing is opportune to take advantage of the new
and reinforced connections between libraries and researchers by
offering new services, or reframing existing curation and preservation
services, around data quality.

% this para no longer seems necessary
%
% Some have argued that the library is the optimal home for digital
% blah...  \parencite{heidorn:libraries} argue that the library is the
% natural place for digital data to then be deposited, preserved, and
% migrated. while I agree that from a consolidation of services
% perspective, it makes sense to have these functions centrally provided
% on campus, the library may or may not be the best service provider --
% central IT, for instance, may have more robust infrastructure around
% information security, identity and access management, disaster
% recovery, high availability, hardware lifecycling, media refresh, and
% so forth. a partnership between the libraries, which have experience
% and expertise in curatorial and preservation practices, and a central
% repository service provider (whether that's the library or not) seems
% most conducive to the long-term viability of research data.

Libraries that lack the resources to sustain a university service
around data quality, or libraries on campuses where other
organizations (such as central IT) might be better resourced or
positioned to provide such services, may play a less active but
equally vital role. Libraries are in large part the centers of campus,
where so much of the institution's research, publishing, and
instruction come together. Librarians that serve as liaisons to
academic departments and research institutes provide a crucial
connection that libraries could use for outreach and marketing in the
area of data quality services; though the libraries may not provide
data quality services themselves, they may serve a consultative role,
pointing at relevant services on campus and abroad, helping to
``knit'' them together for the research enterprise.

Libraries can also offer assistance in the form of instruction, not
radically different from existing information literacy programs,
particularly around practical tools and processes pertaining to
personal digital curation \parencite{williams:lifecycle}. Such instruction
could be especially helpful at institutions where the culture is that
of extreme decentralization or sparse collaboration.

There is a tremendous opportunity as well to offer workshops and
otherwise emphasize the value of curation in providing data quality
for e-science, and also to publicize the ``curation begins before
creation'' mantra. The sooner libraries can insert themselves into the
research process, the better the data quality situation will be on
campus. Libraries need to figure out how to ``hack'' academic culture
and scientific practice in such a way that curatorial skills are
considered required within the new scientific process.

\subsection{It Takes a Village...}

New ``data science'' programs such as the certificate program at the
University of Washington \parencite{uw:datascience} give the author hope
that there is some movement in this area. The focus on data gathering,
analysis, and visualization is an important start; quality and
curation, however, are noticeably absent. A more complete degree
program in data science would effectively combine these topics with
those within data curation and retention, pulling together
domain-specific knowledge, scientific methodology, computer science
techniques, and best practices from the information science,
information technology, and cultural heritage realms to ensure
effective management of data quality over time.

The onus is on cultural heritage institutions such as academic
libraries to make this happen, a daunting and enormous challenge to be
realistic. It falls to us to make a convincing value-added argument
regarding curation and preservation of data to researchers. Funding
agencies like the NSF and NIH can help with this by continuing to
require substantial data management plans, as can academic research
offices and subject disciplines and institutes; forging or
strengthening partnerships with these departments would be strategic
for libraries on campus. This recommendation echoes one of the
findings of the 2006 Association of Research Libraries report on data
stewardship, namely that ``[a] change in both the culture of federal
funding agencies and of the research enterprise regarding digital data
stewardship is necessary if the programs and initiatives that support
the long-term preservation, curation, and stewardship of digital data
are to be successful'' \parencite{arl:stewardship}.

% Not really a good fit anymore.
%
% ``There is a need for a close linking between digital data archives,
% scholarly publications, and associated communication. The potential
% for an expanded role for research libraries in the area of digital
% data stewardship affords opportunities to address these important
% linkages.'' \parencite{arl:stewardship}

\subsection{Our Challenge}

Are academic libraries adequately prepared for this role?  A new suite
of data quality services on campus may require not insignificant
re-skilling and re-education of the workforce, and may also require
some reorganization and redefinition of positions \parencite{jisc:deluge}.

The provision of data quality services and extension of traditional
stewardship to the realm of research data may not be feasible given
the economic environment and existing commitments of academic
libraries. Data quality presents an opportunity to offer new
rationales to University administration for additional funding. The
long-term stewardship risks associated with government-supported
disciplinary data repositories \parencite{baker:funding,merali:peril}
may help make the case that centrally-funded data services protect the
University's investments in new research, increasing return on
investment by ensuring for its long-term stewardship. As research data
is typically owned by the institution itself in the United
States \parencite{dhhs:tutorial,dhhs:guidelines}, and not individual
researchers, it is in the best interest of the institution to take a
proactive role in safeguarding the data.

% No longer a great fit.
%
% ``Knowing that we should not and cannot save everything, librarians
% should apply archival strategies, principles, and practices to
% selection and curation of data. We may start by identifying at-risk
% data that require urgent attention. A survey of our special and
% general collections may discover and expose data we already own amid
% our other holdings. Our current methods of cataloging and metadata
% application will likely need enhancement'' \parencite{ogburn:imperative}
% (archival principles!)

% This is kinda duplicative now.
%
% librarians need the sort of skills to pull this off, etc. (riff on
% this some more, about librarians needing to evolve to manage the
% scalability issues around doing campuswide preservation and curation
% of reserach data.) so too must researchers!  incorporate data mgmt
% into info lit courses, or add new data lit courses for undergrads and
% grads, encourage attendance as part of DMPs?

I agree strongly with Ogburn, who argues that ``funding and planning
for the care and retention of data must be built into the front end,
not the back end, of the research process. Data files must be attended
to while they are compiled and analyzed in order to keep them
available for a reasonable life span. This will require librarians to
be conversant with the language and methods of science, at the table
for campus cyberinfrastructure planning, and working with researchers
at the beginning stages of grant planning''
\parencite{ogburn:imperative}.

Academic libraries need to be conversant with the language and methods
of science and to be involved with advances in campus
cyberinfrastructure. We have the expertise and the challenge of data
quality is well within the traditional mission of libraries. The time
has come for academic libraries to serve as data quality hubs on
campus to enable a new generation of scientific discovery and inquiry
for the good of our society.

\theendnotes

\printbibliography

\end{document}
