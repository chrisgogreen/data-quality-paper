% data-quality-workshop-giarlo.tex
% Michael J. Giarlo's position paper for the Curating for Data Quality workshop
% Author: Michael J. Giarlo
% based upon LaTeX2.09 Guidelines, 9 June 1996
% TODO: Enter date of submission here
% Revisions:	2012-08-xx

% TODO: After having produced the .bbl file (from the .bib
%     file), and prior to final submission, you need to 'insert' your .bbl
%     file into your source .tex file so as to provide one
%     'self-contained' source file.

\documentclass{acm_proc_article-sp}
\begin{document}

% TODO: Revisit this when finished
\title{The Data Quality Layer Cake\titlenote{Paper prepared for 2012
    Curating for Data Quality workshop in Arlington, VA}}

% TODO: Revisit this when finished \subtitle{[Extended Abstract]
% \titlenote{A full version of this paper is available as
% \textit{Author's Guide to Preparing ACM SIG Proceedings Using
% \LaTeX$2_\epsilon$\ and BibTeX} at
% \texttt{www.acm.org/eaddress.htm}}}

\numberofauthors{1} \author{
  \alignauthor Michael J. Giarlo\\
  \affaddr{Penn State University}\\
  \affaddr{E-017 Paterno Library}\\
  \affaddr{University Park, PA  16802}\\
  \email{michael@psu.edu} }
\maketitle

% TODO: Revisit this when finished
\begin{abstract}
  This is a position paper about ensuring data quality for e-science via
  curatorial practices.
\end{abstract}

% TODO: Revisit this when finished
% http://www.acm.org/about/class/how-to-use
% http://www.acm.org/about/class/1998 A category with the (minimum)
% three required fields
\category{H.4}{Information Systems Applications}{Miscellaneous}
% A category including the fourth, optional field follows...
\category{D.2.8}{Software Engineering}{Metrics}

% TODO: Revisit this when finished
\terms{Theory}

% TODO: Revisit this when finished
\keywords{data curation, digital preservation, trust}

\section{Gap Analysis / Challenges}
More about what problems we're elucidating and for which we're proposing
solutions.

Testing out citations in this paragraph to see what the references
section will look like.  First this dude\cite{russom:case} said
something provocative\cite{timmer:faking} about a
topic\cite{uw:datascience} that is near and dear\cite{goble:curation}
to my heart\cite{ogburn:imperative}. Other data\cite{jisc:deluge},
however, suggest otherwise\cite{heidorn:libraries}, which some find
quite troubling\cite{williams:lifecycle}. The absence of
substance\cite{revolutions:litreview} in this paragraph is argued by
experts\cite{iam:assurance} to be deleterious to the
field\cite{wiki:identity}. The vast body of
evidence\cite{wiki:quality} gathered in Never-Never Land, however,
militates\cite{wiki:semweb} against this.


From wikipedia article on Data Quality: ``One industry study estimated the total cost to the US economy of data quality problems at over US\$600 billion per annum (Eckerson, 2002)''

Science is changing -- how can advances in curation practices aid the
practice of good science?

What is data quality?

How is data quality ensured?

trust networks - how are they different now? japanese research example (more
anonymous), galaxyzoo example (much larger). ``quality indicators'' and
``assurance levels''



\section{The Framework}
A framework for understanding data quality is as follows.  data quality stack,
quality as a function of successive indicators

\begin{itemize}
\item Highest quality data is authentic (``good science:'' reliable
  instruments, sound theoretical frameworks, completeness, validity,
  ontological consistency, accuracy)
\item Evaluation of the authenticity of data requires that data be usable
  (curation; domain-specific tools, repos, and conventions; file formats)
\item Usability of data requires that data be accessible, discoverable
  (metadata), identifiable (citation, management (niche for cultural
  heritage orgs)), and associated with sufficient
  documentation/context to use it (curation processes)
\item Data integrity is required for context, usability, authenticity, and
  quality.
\end{itemize}

underlying assumption: data preservation (integrity) is necessary but not
sufficient for data quality. also: context (curation), usability (identity,
discoverability), trust (authenticity). adds up to quality. related to
``Semantic Web Stack''

\section{Angles of Attack}
Recommendations on ways to resolve the problems above.

need better tools (enumerate what makes 'em better) for curation/data
integrity (does ScholarSphere help?)

value of curation, learning that early, using it early in research process -
tie in to crowdsourcing adding value (galaxyzoo) - hack academic culture in
such a way that good curatorial skills are necessary for good science, need
``data science'' programs that marry scientific methodology with data curation
and retention practices

hacking trust networks, new trust networks, more resilient trust networks

% TODO: Revisit this when finished The following two commands are all
% you need in the initial runs of your .tex file to produce the
% bibliography for the citations in your paper.
\bibliography{data-quality-workshop-giarlo}{}
\bibliographystyle{abbrv}

% TODO: Revisit this when finished You must have a proper ".bib" file
% and remember to run: latex bibtex latex latex to resolve all
% references ACM needs 'a single self-contained file'!

% TODO: Revisit this when finished
% http://www.acm.org/sigs/publications/sigguide-v2.2sp (sec 2.8)
% \balancecolumns

\end{document}
